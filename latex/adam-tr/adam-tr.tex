\documentclass[11pt,twocolumn]{article}
\topmargin=0.0in %length of margin at the top of the page (1 inch added by default)
\oddsidemargin=0.0in %length of margin on sides for odd pages
\evensidemargin=0in %length of margin on sides for even pages
\textwidth=6.5in %How wide you want your text to be
\marginparwidth=0.5in
\headheight=0pt %1in margins at top and bottom (1 inch is added to this value by default)
\headsep=0pt %Increase to increase white space in between headers and the top of the page
\textheight=9.1in %How tall the text body is allowed to be on each page

\usepackage{url}
\usepackage{graphicx}
\usepackage{authblk}

\widowpenalty=500
\clubpenalty=500

\date{}

\begin{document}

\title{ADAM: Genomics Formats and Processing Patterns for Cloud Scale Computing}
\author[1]{Matt~Massie}
\author[1]{Frank~Austin~Nothaft}
\author[2]{Chris~Hartl}
\author[1]{Christos~Kozanitis}
\author[1]{David~Patterson}
\affil[1]{Department of Computer Science, University of California, Berkeley}
\affil[2]{The Broad Institute of MIT and Harvard}

\maketitle

\abstract

Current genomics applications are dominated by the movement of data to and from disk. This data movement pattern
is a significant bottleneck that prevents these applications from scaling well to distributed computing clusters. In this report,
we introduce a new set of data formats for genomics applications that are designed for in-memory MapReduce processing.
These formats improve application performance, data storage efficiency, and programmer productivity.

\section{Introduction}
\label{sec:introduction}

The process of transforming reads from alignment to variant-calling ready reads involves several processing stages including
duplicate marking, base score quality recalibration, and local realignment. Traditionally, these stages have involved reading
a Sequence/Binary Alignment Map (SAM/BAM) file, performing transformations on the data, and writing this data back out to
disk as a new SAM/BAM file~\cite{li09}.

\bibliographystyle{acm}

\bibliography{adam-tr}

\end{document}